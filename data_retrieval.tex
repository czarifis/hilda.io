\section{Data Retrieval}
\label{section:dataretrieval}

% Template and template instance figures
\begin{figure*}
\centering
%
%
\begin{minipage}[c]{6cm}
\begin{minipage}[c]{6cm}
\begin{code}
   sources : [ \{ 
     driver   : "postgres", 
     host     : "edu.db.domain", 
     expose   : [ \{
      schema : 'website_info', 
      tables: [visits, page_views] \} ]
     port     : 5432, 
     username : "dbadmin" 
     password : 'myP@ss'
   \}] 
\end{code}
\vspace*{-0.4cm}
\subcaption{DB Access Configuration file}
\vspace*{0.3cm}
\label{figure:source-config-file}
\end{minipage}
%
\begin{minipage}[c]{6cm}
\begin{code}
\textbf{<\% let} readings = 
   SELECT count(time) as visits, time
   FROM (SELECT * FROM page_views pv 
  	     join visitors v 
         on pv.v_id = v.vid) AS joined_table
   GROUP BY time 
   ORDER BY time ASC \textbf{\%>}
\end{code}
\vspace*{-0.4cm}
\subcaption{Data retrieval}
\label{figure:first-running-example:data-retrieval}
\vspace*{0.3cm}
\end{minipage}

\begin{minipage}[c]{6cm}
\begin{code}
readings = [
   \{visits: 15, time: '08:00'\}, 
   \{visits: 10, time: '09:00'\},
   \{visits: 25, time: '10:00'\},  ...]
\end{code}
\vspace*{-0.4cm}
\subcaption{Query Result}
\label{figure:running-example:query-result}
\vspace*{0.3cm}
\end{minipage}
%
\vspace*{0.6cm}
\end{minipage}
\hspace{2cm}
\begin{minipage}[c]{6cm}
\begin{minipage}[c]{7.5cm}
\begin{code}
  \directive{unit}{highcharts} \{
    title: 'Visitor information' ,
    xAxis : \{ 
      labels : ['08:00','09:00'...],
      min : '08:00'
      max : '22:00'
    \}
    series: [\{ data: [ \{y:15\}, \{y:10\}...] \}]
  \} \directive{end unit}{}
\end{code}
\vspace*{-0.4cm}
\subcaption{Unit with evaluated unit state}
\vspace*{0.3cm}
\label{figure:running-example:unit-body}
\end{minipage}
\begin{minipage}[c]{7.5cm}
\begin{code}
  \directive{unit}{highcharts} \{
    title: 'Visitor information',
    type: 'line'
    xAxis : \{ 
      labels : [
        \directive{for}{reading \textbf{in} readings} 
          \directive{print}{reading.time} 
        \directive{end for}{}],
      min : \directive{bind}{min_time},
      max : \directive{bind}{max_time}
    \}
    series: [\{
      data: [ \directive{for}{reading \textbf{in} readings}
          \{
            y  : \directive{print}{reading.count}
          \}
        \directive{end for}{} ]
    \}]
  \} \directive{end unit}{}
\end{code}
\vspace*{-0.3cm}
\subcaption{Template \texttt{temp\_view}}
\vspace*{0cm}
\label{figure:first-running-example:main-template}
\end{minipage}
\end{minipage}
\vspace*{-0.3cm}
\caption{Template, template instance, and UAS configuration file for the running example}
\vspace*{-0.3cm}
\end{figure*}


\remark{It might be a good idea to merge these 3 sections into 1 called ``Walkthrough example'' or something like that. Depends on how large these sections become.}



Data retrieval is one aspect where {\projname} shines. It enables the analyst to write inlined, source specific database queries in order to retrieve data from the respective database servers. In a Python implementation, analysts need to perform a series of tasks before being able to retrieve data. Some of these tasks potentially fall way beyond the skill set of the average data analyst. 

\costas{We should also say that we simplify file imports. For cases, when the data analysts wants to use csv or json files from their local computer, they can use an import button, which is part of our UI, to upload them to the notebook server. After that step is completed we trigger a function that automatically converts this file to JSON and we assign it to a \projname\ variable, so that it can be used in the notebook. Currently, this task requires ssh-ing or scp-ing the files to the notebook server, and potentially moving files to the directive/file-system that is seen by the notebook, which also requires system administrator-linux knowledge}



First, our analyst needs to install and configure database drivers. This step will either introduce a dependency between the analyst and some system administrator, or the analyst will need to have the required access rights in order to perform the installation. The later can be a security concern or can lead to corrupted systems if not performed correctly. The complexity of this step increases as the number of different database systems, our analyst wants to access, increases. More specifically, for cases when an analyst needs to access data stored in a MongoDB and a MySQL database, two drivers will need to be installed.

Once the system is configured, the analyst needs to read lengthy documentation pages and stackoverflow posts in order to properly issue queries to the databases via imperative code by using the library API, then consume the results by using internal constructs of the API, and potentially convert them into a format that assists in data processing. Lastly, the analyst will have to convert the format of the data again in order to feed them into the respective visualization library.

Finally, when implemented in a Jupyter notebook, the analyst's credentials for the database server might lie on plain sight to anyone who has access to the notebook, disrupting the valuable ease of results communication through notebooks.


\costas{Some of these steps, mentioned up to here are common for the data retrieval and visualization stages (e.g installing database drivers and visualization libraries, reading documentation pages to figure out how to use the and so on. Can we group them into a single section for ipython and then explain how things differ in \projname?}

{\projname} addresses each one of the aforementioned imperative code issues efficiently: The analyst generates a configuration file, containing information required for establishing a successful connection with the respective database systems. Figure \ref{figure:source-config-file} shows an example of the configuration file that is used for connecting to a postgres database that contains the data that will by used in our analysis. The configuration file must contain the type of the database system, the host name, port and the credentials that will be used for obtaining a connection. Additionally, it contains the database tables that will be accessible by queries, only the tables that are explicitly defined in the configuration file will be accessible in the notebook. After this configuration file is imported, (which is done directly from the notebook's UI), it will be hidden from the UI and encrypted so that this information is no longer visible to the notebook reader.  Furthermore, as an added benefit, the schema of the accessible tables will be displayed on the UI, this allows the notebook user to get a glimpse of that information when generating the queries. 

Once this step is completed, the analyst is free to issue queries in order to access database systems. Figure \ref{figure:running-example:data-retrieval} contains a sample query that is used in our analysis. The query joins the two tables: $visitors$ and $page\_views$ on the id of the visitor, then groups the result on the $time$ attribute and runs a $count$ aggregate to count the visitors per unit of time. Lastly, it sorts the resulting dataset by time in ascending order. Figure \ref{figure:running-example:query-result} shows the result of the query. Notice that the result has been converted into a JSON array by \projname. In the next section we will show how to use this dataset to generate our first visualization, without the use of any imperative logic.

\subsection{Source Wrappers}
\label{subsection:source-wrappers}


\projname\ contains a set of source wrappers that enables data retrieval from various different sources both relational and non-relational. These source wrappers come pre-installed with \projname\ so that the notebook user will not have to take on any system administrator duties. The user simply provides the configuration file and writes the query in the language supported by each database system. The source wrapper uses that information to connect to the respective database system, retrieve the requested data, convert it into JSON format and make it available to the user. For database systems that do not contain tables with schema, the notebook user has to provide the respective record container under the "expose" attribute of the configuration file (i.e if the user accesses a MongoDB database, she will have to declare a collections of documents instead of database tables). Additionally, in cases when the accessed database system does not have a schema the notebook will simply display a small portion of the dataset, thus assisting the notebook user to compose queries.

\costas{Maybe put the last two sentences  in a side-note?}




