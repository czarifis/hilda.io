\section{Introduction}
\label{section:introduction}

User-friendly data analysis tools and frameworks often provide limited flexibility, as they usually focus on a predetermined set of use/analysis cases or a small fraction of the typically large data analysis pipelines. This lack of flexibility often pushes code-literate analysts towards the use of interactive notebooks such as Jupyter.

Allowing the use of high-level expressive imperative languages such as Python, interactive notebooks are able to assist data analysis, as well as composing results into easily readable notebook-like interfaces. In conjunction with the large number of third party libraries, make interactive notebooks a complete solution for developing, documenting and communicating code and visualizations to other analysts. 

However as we show in this work, interactive notebooks are still suboptimal with regard to ease of use and interactivity. Setting up notebook environments and dependencies, obtaining and combining data and generating the respective visualizations, requires technical knowledge that often exceeds the skill-set of a typical data scientist. Lastly, while such notebooks support the generation of interactive visualizations, this interactivity is not an integral part of the data analysis process. 

We address these issues, by extending interactive notebooks with a template language called {\projname}. The main contributions of this extension are:

\begin{itemize}
	\item \textit{Declarative semantics:} {\projname} implements formal declarative \textit{Model-View-View-Model} (MVVM) semantics. \remark{Fill in why this is a good thing. I have no idea.}
	\item \textit{Expressive template language:} Prior database work, treats a page as a database view. Building on that, our template language goes beyond SQL query and view definition in both style and fundamental expressiveness. It is a mixture of query as well as web templating language that works on ordered (arrays) and semi-ordered (JSON) data. 
	\item We allow in-line declarative code directly in JSON...
\end{itemize}

In this work, we demonstrate the use of {\projname} via a walkthrough example. Specifically, we want to use website access data, plot an access count histogram, as well as the recorder user demographics (age groups). We then want to interact with the histogram plot and select a time region. We want this action to automatically update the second plot with the user demographics in the selected time window. We assume a Jupyter server, where the analysts develop their notebooks and a different database server where data is stored. To retrieve the entirety of the required data, we have to query two different databases and join the returned JSON files. Figure ?? shows how our databases are organized. Our fictional analyst will perform the following tasks:

\begin{itemize}
	\item Data retrieval from remote databases. 
	\item Data curation: Join data and prepare for visualization.
	\item Data visualization.
\end{itemize}

